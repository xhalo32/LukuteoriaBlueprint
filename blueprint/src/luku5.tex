\chapter{Ensimmäisen asteen Diofantoksen  yhtälöt}

\begin{notation}
    $(a, b)$ tarkoittaa $\mathrm{syt}(a, b)$.
\end{notation}

\begin{remark}
    Jos $a, b$ jaetut alkutekijät ovat $p_1, ..., p_l$, voidaan kirjoittaa

    \[
    a = p_1^{\alpha_1} ... p_l^{\alpha_l} \
    b = p_1^{\beta_1} ... p_l^{\beta_l}
    \]
\end{remark}

\begin{lemma}[Yksikäsitteinen jako]
    Jos $a \geq b \geq 1$ niin on olemassa yksikäsitteiset $k$ ja $r < b$, joille

    \[
        a = k b + r.
    \]
\end{lemma}
\begin{proof}
    Huomataan, että $\exists r < b, a = k b + r$ on ekvivalentti sen kanssa, että $k b \leq a \lt (k + q) b$ kaikilla $k$.

    $a$ sijoittuu joidenkin lukujen $b, 2b, 3b, ...$ väliin jollakin $k$, eli $k b \leq a \leq (k + 1) b$.
\end{proof}

\begin{lemma}
    $a \gt b \geq 1$, silloin $(a, b) = (a - b, b)$.
\end{lemma}
\begin{proof}
    Olkoon $d := (a, b)$ ja $d' := (a - b, b)$.
    Tiedetään, että $d \mid a$ ja $d \mid b$, joten $d \mid (a - b)$.
    Tästä saadaan $d \leq d'$.
    
    Toisaalta, $d' \mid b$ ja sillä $d' \mid a - b$ niin $d' \mid a$.
    Tästä saadaan $d' \leq d$.
\end{proof}

\begin{remark}
    Jos $a = k b + r$ niin

    \[
    (a, b) = (a - k b, b) = (r, b) = (b, r)
    \]
\end{remark}